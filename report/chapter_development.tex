\chapter{Development}~\label{cha:development}
\section{Frontend}
\subsection{Implementation}
The actual implementation of the frontend used Vite as a build tool for ReactJS. This gave a starting TypeScript template which was modified to create the application.

Once the build system was in place the design was broken up into components so each to be built independently. The HeroUI component library used in this project provided basic components common to most web applications, such as buttons, modals, and input fields from these the more complex components could be built. One of the benefits of using this component library was its integration with Tailwind CSS, which allowed for stylings to be applied through class names rather than bespoke stylesheets. This greatly reduced the amount of time needed to style the application and ensured a consistent look across the application.

Once the components were built to use the application the next step was to integrate with the backend services. For some of these to work the user must authenticate with Spotify to gain access to the main functions of the Spotify API. This had to be implemented on the frontend using Spotify's OAuth2 API which allowed for the user to authenticate with Spotify and then be redirected back to the application with an access token in the URL which could be used to interact with the Spotify API.

Once this was complete the rest of the API endpoints for the backend could be implemented. This was done using the Axios library.

\subsection{Play Screen}
The play screen mostly follows the initial design, comprising three main sections: the track list, the controls, and a spinning vinyl animation. Music playback is handled by the Spotify Web Playback SDK, a client side JavaScript library that enables control of Spotify playback through the browser. For actions such as changing tracks, skipping forward or backward, and pausing, the application calls a method provided by the SDK.

\paragraph{Track List} The track list is a list of all tracks the currently active album, which is set on the scan screen. It presents users with a selection menu of tracks which can be selected in order to play a specific track.

\paragraph{Controls} The playback controls include buttons and sliders which effect the playing music in some way. Clicking a button or adjusting a slider invokes the relevant SDK methods, allowing users to play, pause, skip, or adjust volume. Album art and track information automatically update as the active song changes. There is one feature which was removed from the original design, the ability to scrub through the whole album rather than only the currently playing track was considered too complex to implement and added little to the user experience and so was removed.

\paragraph{Spinning Vinyl} To replicate the feel of a physical record player, a simple animation was implemented to rotate a vinyl graphic which had the album cover in the centre. Just as if this were a real record player, the vinyl rotates when music is playing and stops when paused. This was controlled by editing the rotation styling of the graphic. Another animation was added to simulate the act of removing the vinyl from its sleeve when the album is changed. This was done by sliding the album art of the screen as if the vinyl was being removed from it.

[TODO: Add Figure of the play screen]

\subsection{Scan Screen}
The scan screen, though following the initial design as closely as possible, had to be changed during development as some problems with the original design became clear. It was also split into three main parts like the play screen: the album confirmation sidebar, the scan section, and the album collection.

\paragraph{Album Confirmation Sidebar}
The album confirmation sidebar existed to allow the user to confirm the album that was guessed was the correct one. This presented a problem that was not considered in the original design, what if the album guessed was incorrect? To solve this a second stage of the confirmation with a separate scatter shot approach taken to find the correct album [TODO: Add figure]. The sidebar also acts as a way to show to the user the application is processing as whilst the application is determining the album the sidebar is animated to show a loading animation providing feedback to the user that the application has not broken. Once the user confirms the album selection the sidebar slides back to the side of the screen and the album is set as the active album.

\paragraph{Scan Section}
Similarly to the confirmation sidebar an issue was found with the scan section. The original design did not take into account users who may not have a camera connected to their device. To solve this a file upload was added to the scan section which is the default if no cameras can be accessed by the browser [TODO: Add Figure]. This could also be accessed if the user did not want to use their camera for any reason.

\paragraph{Album Collection}
This section of the screen was mostly unchanged from the original design. The user's albums slide across the bottom of the screen and the user can select an album to play by clicking on it [TODO: Add Figure]. Once the user hovers over an album art the sliding is paused and extra details of that album appear such as the title and artist. The option to look through the user's collection differently is also available by clicking the view all button which opens a modal with all the user's albums in a view reminiscent of flicking through vinyl records in a record store.

[TODO: Add Figure of the scan screen]

\subsection{Social Screen}
The social screen had the most changes from the design. It was originally implemented as shown in Figure~\ref{fig:social_screen_mockup}, but this proved to be too cluttered with larger collections being increasingly difficult to look through. The final design was inspired by the act of flicking through vinyl records in a record store. Each user's collection appears as a section of a grid where the albums in that collection pop up and down at random intervals. This allows for many more collections to be shown on the screen at once without overloading the user with information. The user can also load more collections by clicking the load more button at the bottom of the screen and view collections in more detail by clicking on them.

[TODO: Add Figure of the social screen]

\subsection{Automated Testing}
Automated testing of the frontend was done using Vitest a testing system built for projects using the Vite build system. A test driven development approach was taken so prior to starting the creation of each component a test case was written for each feature the component should have. Then the component was written that could pass those tests. In some cases the test cases found to not be perfect and so some refactoring was needed to make the tests pass. This was a good way to ensure that the components were built to the specification and that they could be tested in isolation.

The goal for the automated testing was to reach a coverage of 95\% which would provide certainty that the application was working as expected. To ensure this a GitHub action was set up to run the tests on every push to the repository and to fail the build if the coverage was below 95\% or if a build failed.

\subsection{Containerisation}
Containerisation for the frontend involved creating a Docker image which contained a built version of the frontend that on start would serve the application with an exposed port that users could connect to.

The building of the application was straightforward after dependencies are installed however these dependencies are not required to be in the Docker image once the build is complete and since cloud storage has a cost associated with it the size of the image should be minimised. To do this a two stage build is employed with the first stage being the build stage where the application is built and the second stage being the production stage where the built application is copied from the build stage and served.

\subsection{Dependency Management}
Dependencies for the front end were managed using Node Package Manager (NPM). As new dependencies were installed a file in the repository would be updated with the name of the package and the version. From this file anyone with NPM could install the dependencies by running a single command.

Automatic scanning of the dependencies file was done using Dependabot which would check for updates to the dependencies at a set interval, such as weekly or monthly, and create a pull request with the updates. This meant all dependencies were kept up to date and security vulnerabilities were patched as soon as possible whilst the automated tests allowed easy verification of if new dependency versions broke any functionality.

\subsection{Challenges}
\subsubsection{Environment}
TODO: Add details about the problems I have with configuring environments

\section{Backend}
\subsection{Implementation}
\subsubsection{Changes from initial design}
\subsection{Automated Testing}
\subsection{Containerisation}
\subsection{Dependency Management}

\section{Code Quality} \label{sec:code-quality}
