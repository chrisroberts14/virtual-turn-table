\chapter{Development}~\label{cha:development}
\section{Frontend}
\subsection{Implementation}
The actual implementation of the frontend used Vite as a build tool for ReactJS. This gave a starting TypeScript template which was modified to create the application.

Once the build system was in place the design was broken up into components so each to be built independently. The HeroUI component library used in this project provided basic components common to most web applications, such as buttons, modals, and input fields from these the more complex components could be built. One of the benefits of using this component library was its integration with Tailwind CSS, which allowed for stylings to be applied through class names rather than bespoke stylesheets. This greatly reduced the amount of time needed to style the application and ensured a consistent look across the application.

Once the components were built to use the application the next step was to integrate with the backend services. For some of these to work the user must authenticate with Spotify to gain access to the main functions of the Spotify API. This had to be implemented on the frontend using Spotify's OAuth2 API which allowed for the user to authenticate with Spotify and then be redirected back to the application with an access token in the URL which could be used to interact with the Spotify API.

Once this was complete the rest of the API endpoints for the backend could be implemented. This was done using the Axios library.

\subsection{Play Screen}
The play screen was implemented mostly as designed with the ability to play, pause, skip, scrub through a song, and select individual tracks from the currently active album [TODO: Add figure]. The spinning vinyl was implemented using an animation that controlled its rotation angle and pausing when playback was paused.

\subsection{Scan Screen}
The scan screen had some changes from the initial design. Originally the intention was to connect to the users camera and display the stream on the web browser. This was done but the case of when the user does not have a camera was overlooked. To fix this a file upload was added and is the default if no cameras can be accessed by the browser [TODO: Add Figure]. Along with this the confirmation sidebar changed to allow for cases where the original first guess was incorrect, and a scatter shot approach is instead used to find the correct album. This was done by allowing the user to select from a selection of possible albums [TODO: Add Figure].
The album selection seen at the bottom is still the same as designed with the suers albums sliding across the bottom and the user being able to select an album to play by clicking on it [TODO: Add Figure].

\subsection{Social Screen}
The social screen had major changes from the initial design. The original design had columns of albums for each user. In practice this meant that very few user collections could be shown at once without overloading the screen. The approach taken here was to instead take an approach that resembled the act of flicking through vinyl records in a record store. This minimised the amount of space each user's collection needed so many more could be shown. There was also the addition of a load more button, in order to prevent the page from taking too long to load or becoming too cluttered the page only loads with a subset of the collections that could appear. The load more button allows the user to load more collections as they scroll down the page [TODO: Add Figure].

\subsection{Automated Testing}
\subsection{Containerisation}
\subsection{Dependency Management}
\subsection{Challenges}
\subsubsection{Environment}
TODO: Add details about the problems I have with configuring environments

\section{Backend}
\subsection{Implementation}
\subsubsection{Changes from initial design}
\subsection{Automated Testing}
\subsection{Containerisation}
\subsection{Dependency Management}

\section{Code Quality} \label{sec:code-quality}
