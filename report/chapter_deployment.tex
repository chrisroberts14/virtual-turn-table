\chapter{Deployment}~\label{cha:deployment}
Once development was complete, the application had to be deployed to be accessible to users. This involved configuring two ways to run the application: locally, using Docker Compose and on the web, using a cloud hosting service
TODO: Add more content here.

\section{Docker Compose}
Docker Compose is a container orchestrator that allows multiple containers to be run together on one system in unison [TODO: Add ref here]. Though many IDEs support running multiple containers at once, Docker Compose replicates the production environment more accurately.
% TODO: Add More

\section{On the web}
The web deployment was done using Google Cloud Run. A configuration can be pointed to the Dockerfile in the repository and, once triggered by an event on the repository, will build the image and deploy it to the web. It provides a URL and handles the complexities of hosting such as spinning up instances when needed and shutting them down when not in use.

The intention was for a build to be conducted and deployed after each push to the main branch, assuming every push passes all scans and tests. This, however, was deemed to be excessive for this application and so a tag mechanism was implemented, so builds are only triggered when a new tag is pushed to the repository.

This service, however, comes at a monetary cost as this is an infrastructure as a service (IaaS) model. Higher usage would result in higher costs especially for the storage of the Docker image after it is built on the cloud servers. For this project the free tier proved to be sufficient but if the application were to be used more widely costs would need to be considered.

%Cite some references \cite{dyke1994, latex2012}
